\section{Acknowledgements / Preface}
This piece of software is for a large part the result of the work on my Msc. thesis in 2014-2015 on non-linear time series analysis of groundwater levels. One of my goals during my thesis-period was to learn programming in python, a goal easily met with Mark Bakker as my supervisor. Somewhere halfway through the process, I realised how much time and effort had gone into programming, and the idea floated in my head of writing a python based software program that I could later share with others. As I think educating yourselves is part of your thesis, I took the opportunity to learn how to develop a software program and share it through Github. It meant that others would have to be able to read, work and make changes to my scripts. 'Acting' as if I was not going to be the only user of your script, really changed to way I did my programming.

However, apart from these personal incentives, I also had more fundamental reasons for starting this project. Firstly, as far as I am aware, there are no programs available that are completely open-source. Menyanthes is available on a commercial basis, and the Groundwater Statistics Toolbox by Peterson et al. [2014] requires Matlab. Secondly, you need complete control over the modelling process, which in my opinion is necessary for scientific work. While other software provides a graphical user interface (GUI), fast optimization and a great user experience, it is very difficult to change parameters, underlying assumptions or model structure. Especially the latter is important, as non-linear groundwater time series analysis requires a flexible model structure to implement the hydrologists system knowledge into the model, as I advocated in my thesis. Finally, the methods that I used in my study can be applied to other models without the user needing to write this piece of code all over again. 

GWTSA is an object-oriented program written in python. It allows the user to perform time series analysis of groundwater levels with just a few lines of code (Chapter3). Information on the model, its performance and the parameter estimation is easily accessed (chapter 4). However, the object oriented program structure also allows you to test new model structures, and a guide for adding new model structures is given in chapter 5. The software is envisioned for practical use, but also for more scientifically oriented projects. The scientific concepts that sit behind the model are discussed in-depth in chapters 6 and 7, as well as how these concepts are implemented in the model (chapter 8). It is tried to make to software as transparent as possible, so if anything is unclear don't hesitate to contact the author!

R.A. Collenteur [\url{info@raoulcollenteur.nl}]



 