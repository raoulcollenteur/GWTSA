\chapter{Installation}

\subsection{Where to get GWTSA?}


\subsection{Installing on a Mac}

\subsection{Installing on Windows (Experimental)}

\subsection{Installing on other Operating Systems}

\subsection{Compiling the Unsaturated Zone Module using Cython}
The GWTSA software is unique in the point that it offers the ability to use a non-linear model to calculate the recharge. However, this comes at a price, most notably in terms of model complexity and computation times. The unsaturated zone model (captured in the python and cython files unsat\_zone.py/unsat\_zone.pyx) can increase the computation times dramatically and is therefore ported to a compiled language for increased performance in terms of computation speed. There is an interesting python package that can help in porting python code to compiled C-code: Cython (\url{http://cython.org}).

Cython can port pure python scripts to compiled C-code, but the speed-up will generally be limited to a few orders of magnitude (1-3 times as fast). A few modifications to your python script however, can eliminate bottlenecks and really boost the performance of your code. This cython-enhanced version of our python script is unsat\_zone.pyx, and is ready to be 'cythonized'. Cythonizing is the process of compiling your cython script to C-code and something that can than be imported back into your python scripts. On a mac, this means you create a Shared Object file, (unsat\_zone.so), and on windows you will create another file type (unsat\_zone.pyd). This means that the file extension is dependent on your operating systems and compiling the file needs to happen on the same operating system that GWTSA is used on. Compiling is straight forward and can be done following the following steps.

\textbf{On Mac OSX:}
\begin{itemize} 
\item{Open Terminal}
\item{Move to the directory where GWTSA is located using the 'cd' command (e.g. 'cd Projects/GWTSA')}
\item{Type 'python setup\_unsat\_zone.py build\_ext --inplace' and press Enter}
\item{Cython now compiled the code and a .so file is created. When available this .so file is automatically imported by python when importing GWTSA (Python import has a preferred order promoting compiled scripts if available) }
\item{\textbf{!!} It sometimes happens that the compiled .so file is put in a folder within the GWTSA folder, you should then move the unsat\_zone.so file to the GWTSA folder}
\end{itemize}

\textbf{On Windows:}
\begin{itemize} 
\item{Open command window (type 'CMD' in the start menu)}
\item{Move to the directory where GWTSA is located using the 'cd' command (e.g. 'cd Projects/GWTSA')}
\item{Type 'python setup\_unsat\_zone.py build\_ext --inplace' and press Enter}
\item{Cython now compiled the code and a .pyd file is created. When available this .pyd file is automatically imported by python when importing GWTSA (Python import has a preferred order promoting compiled scripts if available) }
\item{\textbf{!!} It sometimes happens that the compiled .pyd file is put in a folder within the GWTSA folder, you should then move the unsat\_zone.pyd file to the GWTSA folder}
\end{itemize}